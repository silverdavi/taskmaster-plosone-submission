\documentclass[11pt]{article}
\usepackage[margin=1in]{geometry}
\usepackage{times}
\usepackage[hyphens]{url}
\usepackage[colorlinks=true, allcolors=blue]{hyperref}
\usepackage{enumitem}

% --- Custom Environments for Professional Formatting ---
% Environment for quoting the reviewer's/editor's comment
\newenvironment{comment}{%
  \begin{quote}\em\noindent\ignorespaces%
}{%
  \end{quote}%
}

% Environment for your response (changed to "My Response")
\newenvironment{response}{%
  \begin{quote}\noindent\textbf{My Response:}\par\medskip\noindent\ignorespaces%
}{%
  \end{quote}%
}
% --- End of Custom Environments ---

\begin{document}

% --- Title Block ---
\begin{center}
{\Large \textbf{Response to Reviewers}}\\
\vspace{0.3cm}
{\large \textbf{PONE-D-25-28524}}\\
{\large Investigating Audience Preferences Within the Hybrid Competitive-Comedic Format of Taskmaster UK}\\
\vspace{0.3cm}
David H. Silver\\
\textit{Submitted to PLOS ONE}\\
\vspace{0.3cm}
July 23, 2025
\end{center}

\vspace{0.5cm}

% --- Opening Salutation ---
Dear Dr. Scaramanga and Dr. García-Béjar,

Thank you for your time and for providing constructive feedback that has strengthened my manuscript. I have revised the paper to address all points raised by the editor and reviewer. Below, I detail the changes made in a point-by-point format.

\vspace{0.5cm}

% --- Response to Editor ---
\section*{Response to Editorial Comments}

\begin{enumerate}[label=\textbf{Editor's Point \arabic*:}, wide, labelwidth=!, labelindent=0pt, leftmargin=*]

    \item \textbf{Style Requirements, File Naming, and Figure References.}
        \begin{comment}
        “Please ensure that your manuscript meets PLOS ONE's style requirements... Please ensure that you refer to Figures 2, 4, 5, 6, 7 ,8, 9, 10, 11, and 12 in your text...”
        \end{comment}
        \begin{response}
        I have reformatted the manuscript to meet all PLOS ONE style requirements (PACE). I have also performed a thorough check to ensure every figure in the manuscript is now explicitly referenced in the main text.
        \end{response}

    \item \textbf{Copyright for Map Image (Figure 5, now Figure 4).}
        \begin{comment}
        “We note that Figure 5 in your submission contain [map/satellite] images which may be copyrighted... We require you to either (a) present written permission... or (b) remove the figures from your submission.”
        \end{comment}
        \begin{response}
        I have addressed this point by explicitly stating that the map is published under a Creative Commons Attribution-Share Alike 4.0 International license. I have included the proper attribution in the figure caption as required.
        \end{response}

    \item \textbf{Placement of Supplementary Figures.}
        \begin{comment}
        “We notice that your supplementary figures are included in the manuscript file. Please remove them and upload them with the file type 'Supporting Information'.”
        \end{comment}
        \begin{response}
        The supplementary figures have been removed from the main manuscript file. They have been prepared for separate upload as 'Supporting Information' files, each with a corresponding legend listed after the references in the main text.
        \end{response}

\end{enumerate}

% --- Response to Reviewer ---
\section*{Response to Reviewer \#1 (Dr. Ligia García-Béjar)}

I thank the reviewer for their constructive comments and suggestions. The reviewer's own publication on Netflix viewership helped me identify relevant works that sharpened the relevance and theoretical grounding of the literature review in my manuscript.

\begin{enumerate}[label=\textbf{Reviewer Comment \arabic*:}, wide, labelwidth=!, labelindent=0pt, leftmargin=*]

    \item \textbf{Strengthen Connection to Existing Literature and Theory.}
        \begin{comment}
        “Perhaps it would benefit from a more explicit review of the literature and theory to implicitly justify the topic chosen and how the methodology is relevant to answering the research questions.”
        \end{comment}
        \begin{response}
        I have expanded the Introduction with a new literature review that grounds the study in established theoretical frameworks. This section now incorporates:
        \begin{itemize}
            \item \textbf{Uses and Gratifications Theory} to frame audience motivations.
            \item \textbf{Parasocial relationship} and engagement spectrum theories (e.g., Hill, 2017; Enli, 2012).
            \item Theories of \textbf{hybrid entertainment formats} and international format adaptation (which are highly relevant to the Taskmaster case study).
            \item \textbf{Cross-cultural perspectives} on media consumption and cultural proximity.
        \end{itemize}
        This theoretical foundation now explicitly justifies the hypotheses and methodological choices.
        \end{response}

    \item \textbf{Add Practical Industry Implications.}
        \begin{comment}
        “I also suggest that the conclusions include recommendations for the industry.”
        \end{comment}
        \begin{response}
        I have added a new subsection to the Conclusions titled \textbf{“Implications for Entertainment Industry Practice.”} This section translates my empirical findings into actionable guidance for producers and casting directors on:
        \begin{itemize}
            \item Casting strategies to consider performer age and experience.
            \item Managing emotional tone, specifically "awkwardness."
            \item The role of competition design versus performer chemistry.
            \item Structuring series pacing and adapting formats for international markets.
        \end{itemize}
        \end{response}

\end{enumerate}

% --- Concluding Remarks ---
\vspace{1cm}

These revisions address all feedback provided. Thank you for your time and consideration.

\vspace{0.5cm}

Sincerely,

\vspace{1cm}

David H. Silver\\
Remiza AI\\
\href{mailto:david@remiza.ai}{david@remiza.ai}

\end{document}